\subsubsection{Performance Requirements}

	\paragraph{}
	\mbox{}\\
	One of the major requirements for this system is how quickly the system should respond to requests, 
	otherwise known as response time. Expanding further, it is simply maximising a feasible satisfactory
	time by which a system should or could respond, based on the kind of system, and more, in a 
	user-computer/device setting. 

	When a user interacts with an application in general, a longer response from the system may cause 
	the user to think that the system is down in one way or the other. In addition, response time degradations 
	can cost so much depending on the time of the day the system is most used (e.g. during peak hours on UP’s 
	Main Campus, with respect to the NavUP System). 
	Therefore, the system ought to adhere to protocols that would increase response time at any given,
	whether there are many individuals actively on the system vs vice versa. Users should not have to 
	face the issue of observing system lags as such, which in essence would get them disinterested in
	using the application as a whole. 

	In addition to the above, the system should also engender throughput, which is the amount of data 
	a system can handle per time or the workload at which the response time is to be met. With about 
	30000 people going in and out of UP’s Main Campus every day, NavUP’s system should be made to approximate 
	how many users would  be logged to their system per day and per time and be able to accommodate the 
	traffic that will come through to the system. 
	When the application launches, there will automatically be high traffic flooding into the system, 
	as almost everyone will be a new user. Thus the amounts of data to be managed per time and per quantity 
	should be in place to avoid any forms of system lagging, or high traffic levels to the point where new users 
	can’t register at all or other users can’t even access their accounts, log in or even search for a location
	to navigate to, which would yield frustration for the user and thus discouraging them entirely,
	as earlier mentioned. 

	For smooth user experience, back-end technicalities should also not interfere with the user interface,
	and therefore items like the networks of the system should be kept at bay.

	In essence, this system should be able to be as responsive as possible to the user with minimal to 
	no lags, it needs to handle the capacity of the plus minus 30000 users that may continuously log on 
	to the system. It should also be able to safely store information from this large user space (30000) to 
	the system, and retrieve them in a timely manner (response time).

	Relating to accounts, the system ought to allow concurrency in the creation of user accounts and the
	accessing of these profiles as well. Data usage should be kept to a minimum, allowing streams to be 
	well managed. 

	Lastly, during search and navigation on the application, performance ought to be efficient (real-time search), 
	with data updated and arriving as required.

