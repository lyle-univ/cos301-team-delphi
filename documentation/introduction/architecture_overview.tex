The system architecture can be broken up into 4 high level component. The mobile interface, the web interface, the core system and agents that do work on data from the core system. Each of these high level component can be broken up into smaller subsystem. These subsystems are enumerated and their place in the system is explained with the diagram below.

\begin{figure}[h]
    \makebox[\textwidth][c]{\includegraphics[width=1.4\textwidth]{diagrams/layered_architecture.pdf}}
\caption{Architecture Overview}
\end{figure}
There is a lot going on in this diagram, so let us start with some of the conventions that we used to communicate how different subsystems relate to each other. The Dark blue is everything relates to GIS data. The purple is all data that gets persisted in the database. The Yellow is to show that add-ons exist on more than one of the user interfaces, namely Mobile and WebUI. The green is everything relates to add-ons. The Red is everything that relates to the position information of mobile devices. The Orange is everything that relates to user management. You will notice that the WebUI has an Administration component that is Orange, which manages anything user related in the User Management subsystem. Grey is everything that relates to agent, for example agents can be observers.
