\subsubsection{Scope}
\paragraph{}
\mbox{}\\
The navigation module of this system consists of providing services such as 
calculating routes for a user and directing one from a specific point to another, 
as a user would have specified to the system’s input. The system is to be able 
to get, delete, save, modify and record routes, and effectively and efficiently steer 
a user to their destination, thus establishing the desired results.In addition to this, 
the system will endeavour to maintain reliability, robustness, high cohesion, 
low coupling, efficiency, low complexity, feasible maintainability and good portability. 


\subsubsection{Software Attributes}

	\paragraph{Efficiency}
	\mbox{}\\
	The navigation subsystem will give realtime directions and pathfinding with minimal 
	expenditure of time and resources. Some of these resources include battery power and mobile 
	data. The decision of what route to take should be calculated with little to no latency and 
	should adhere to all the conditions set by the user. The system should also require minimal 
	memory, be small in size and produce optimal performance whilst taking into consideration 
	time and resource constraints.

	\paragraph{Portability}
	\mbox{}\\
	The navigation system should be integrated in such a way as to ensure ease of portability
	 between the web interface and the different devices that could host the mobile application. 

	\paragraph{Maintainability}
	\mbox{}\\
	It will allow for the evolution of the system by easily saving routes and if the route is 
	popular enough, incorporating it into the default routes used. The system will be modular
	 enough that separate sections can be modified and improved with minimal effect on the 
	 system as a whole. Ideally faulty or worn-out components can be repaired or replaced
	  without having to replace still working parts, this will also allow for isolation of
	  defects in the code and their subsequent correction.
	
	\paragraph{Cohesion}
	\mbox{}\\
	This system will have high cohesion since all functions will be directed towards the core
	 function of directing the user from one location to another. Each function will be
	  strongly related and will work together as a functional unit.
	  
	\paragraph{Reliability}
	\mbox{}\\
	This system should always be able to find the shortest route as well as the simplest route 
	to get to the desired location. It will always find locations that are accessible and vice 
	versa, always giving feedback. In addition, information that needs to be synced should be done
	frequently, so as to adequately and concisely direct to the chosen location based on accurate
	information. 
	
	\paragraph{Robustness}
	\mbox{}\\
	This navigation system will not be easily breakable as it will be error-proof to a feasible 
	extent, to avoid unnecessary faults and not wholly be affected by single application failures. 
	The system should be able to recover quickly from such failures, or at least be able to 
	hold up, or return to a valid stage or state.
	
	\paragraph{Complexity}
	\mbox{}\\
	This system should be as simple as possible, enabling users to easily find routes, select routes,
	save routes, and subsequently follow their chosen routes without being misled or lost in any way.  
	After choosing desired routes, this system should allow users to easily modify routes. Thus it should
	endeavour to be flexible with users to a feasible point.
	
	\paragraph{Coupling}
	\mbox{}\\
	Although this subsystem will be slightly dependent on some core systems such as GIS, this system
	should aim to incorporate loose coupling, being able to be as stand-alone as is necessary. 
	That is, this system should be able to, for example, provide routes to the user with information 
	pre-streamed to the system in case of any mishaps (i.e. offline functionality incorporated),
	or be able to give directions without a user having an account, with details to be relayed to and 
	fro from the DBMS (i.e. Guest users).
	
  
  
\subsubsection{Technology choices}
	The Anyplace API will be used as the naviation service for this sub-system. Anyplace is a good 
	choice in API for a number of reasons: Anyplace is a free API hosted on GitHub and created by 
	university students. It provides indoor and outdoor navigation with accuracy of 1.96 meters. 
	This is perfect for NavUP since it uses Wi-Fi to navigate and provides floormaps, which can be 
	used in the GIS module. It also has additional features such as Wi-Fi heatmaps which will 
	supliment other modules. This API allows for fast accurate addition of new buildings and it 
	supports point of interest(PIO) to PIO navigation (plus easy addition of POI`s). On top of all 
	this it also has support for crowdsourcing. This API contains many of the main features that NavUp
	 intends to have, plus a number of extra features that will enhance NavUp`s navigation abilities,
	  making it a good technology choice for the navigation module.
\pagebreak
\subsubsection{Use Case Diagram}
	\begin{figure}[h!]
	    \makebox[\textwidth][c]{\includegraphics[width=1.4\textwidth]{diagrams/navigation_subsystem/navigation_subsystem_draft.pdf}}
	\caption{Navigation Use Case}
	\end{figure}
\pagebreak
\subsubsection{Class Diagram}
	\begin{figure}[h!]
	   \makebox[\textwidth][c]{\includegraphics[width=1.4\textwidth]{diagrams/navigation_subsystem/Navigation_subsystem_class_diagram_With_DesignP_draft.pdf}}
	\caption{Navigation Class Diagram}
	\end{figure}

\subsubsection{Design Patterns}
	Both memento and strategy design patter will be used for this subsystem. Memento will persist the saved/recorded routes for later use and strategy will decide how the route will be calculated.

