\subsubsection{Scope}
\paragraph{}
\mbox{}\\
As per the specification user management will be invloved with the creation of users, specifically the creation of different types of users. The users will follow a hierarchy approach with each higher level having more privileges than the previous. The heirachy, from top to bottom, is as follows: Admin, User, Guest. Each type of user will fill a specfic role and purpose. The admin being able to manage and create other users as well as the ability to update location information and set points of interest. The standard user will a student or employee usings the app. Lastly the guest user, which will not require a profile, will be used by outsiders or those that are trying the app.

\subsubsection{Software Attributes}
\paragraph{Security}
\mbox{}\\
The user data needs to be secure in order to protect the users. By implementing a secure database in which user data is stored as well as a firewall to protect the server itself we can ensure that the system will remain secure. User passwords will be secured using the bcrypt hashing algorithm and the tuning parameter will be adjusted to balance both security and performance requirements. Due to some of the information stored in the user files being somewhat sensitive data (routes walked and common locations visited) it is crucial for us to keep these records secure and safe.

\paragraph{Portability}
\mbox{}\\
The system is inherently designed to be very portable. The user data can be entered regardless of the hardware device being used. This means that users can register on iPhone or Android. Admin users can create users if needed directly from the server. The software independence is also clear as again any device,  rergardless of thier operating system, can be used to access and manage the user data. 

\paragraph{Cohesion}
\mbox{}\\
The system is already a very coherent system. The user data will have to be used across a lot of the different subsystems such as Navigation and Rewards and as such it needs to be designed to be portable. To do this the user subsystem will include a variety of functions to return the required data need by another subsystem. 

\subsubsection{Technology Choices}
\mbox{}\\
For the users subsystem we will need a server to host the relevant tables. This server should have a database management software to allow for easy management. The server should be 
guarded by some form of firewall software to prevent any unwanted access. 

\subsubsection{Use Case Diagram}
	\begin{figure}[h!]
	\makebox[\textwidth][c]{\includegraphics[width=1.4\textwidth]{diagrams/users_subsystem/users_subsystem.pdf}}
	\caption{Users Use Case}
	\end{figure}
	
\subsubsection{Class Diagram}
	\begin{figure}[h!]
	\makebox[\textwidth][c]{\includegraphics[width=1.4\textwidth]{diagrams/users_subsystem/users_class_diagram.pdf}}
	\caption{Users Class Diagram}
	\end{figure}

\subsubsection{Design Pattern Diagram}
	\paragraph{Reasoning}
	\mbox{}\\
	We chose the factory desgin pettern as it will allow us to stream line the process of creating users. 
	\begin{figure}[h!]
	\makebox[\textwidth][c]{\includegraphics[width=1.4\textwidth]{diagrams/users_subsystem/users_factory_design_pattern.pdf}}
	\caption{Users Class Diagram}
	\end{figure}
